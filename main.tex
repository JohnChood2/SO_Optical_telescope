\documentclass{article}
\usepackage{tabularx}
\usepackage{multirow}
\usepackage{amsmath}
\usepackage{geometry}
\usepackage[table]{xcolor}
\usepackage{hyperref}
\usepackage{enumitem}
\usepackage{fancyhdr}
\usepackage[utf8]{inputenc}
\usepackage{graphicx}
\usepackage{booktabs}

% Page layout settings
\geometry{a4paper, margin=1in}

% Header/Footer settings
\pagestyle{fancy}
\fancyhf{}
\rhead{PARQUE ASTRONÓMICO ATACAMA}
\lhead{1-meter Telescope Proposal}
\cfoot{\thepage}

\title{Project Proposal: 1-meter Class Optical Telescope\\
PARQUE ASTRONÓMICO ATACAMA}
\author{Dr. John C. Hood II}
\date{October 1, 2024}

\begin{document}

\maketitle

\tableofcontents

\section{Executive Summary}
This proposal outlines the establishment of a 1-meter class optical telescope at the PARQUE ASTRONÓMICO ATACAMA in Chile. The project addresses a critical gap in the astronomy community's observation strategy: the lack of high-cadence optical observations of AGN and transients in the southern hemisphere. This facility will provide valuable data for both internal analysis and public access through an open database.

\section{Project Overview}

\subsection{Scientific Motivation}
Recent advances in millimeter-wave astronomy have revealed a significant need for complementary optical observations of AGN and transients in the southern hemisphere. High-cadence optical monitoring of these sources is essential for:
\begin{itemize}
    \item Multi-wavelength correlation studies
    \item Rapid follow-up of transient events
    \item Long-term variability monitoring
    \item Public data access for the broader astronomical community
\end{itemize}

\subsection{Site Selection}
The PARQUE ASTRONÓMICO ATACAMA presents ideal conditions for astronomical observations:
\begin{itemize}
    \item Exceptional atmospheric transparency
    \item Minimal light pollution
    \item High altitude benefits
    \item Existing infrastructure accessibility
    \item Security through proximity to other facilities (CLASS, APEX, ALMA)
\end{itemize}

\section{Technical Specifications}

\subsection{Primary Observatory Components}

\subsubsection{Telescope System: Planewave CDK1000}
\begin{itemize}
    \item Aperture: 1000mm (1 meter)
    \item Focal Length: 6,800mm (f/6.8)
    \item Mount: L-500 Direct Drive
    \item Pointing Accuracy: Sub-arcsecond with TPoint model
    \item Maximum Slew Speed: 20 degrees/second
\end{itemize}

\subsubsection{Imaging System: ZWO ASI183MM}
\begin{itemize}
    \item Sensor: Sony IMX183CLK-J/CQJ-J CMOS
    \item Resolution: 20.18 Megapixels (5496 x 3672)
    \item Pixel Size: 2.4 µm
    \item Cooling: 40°C below ambient
    \item QE Peak: 84% at 550nm
\end{itemize}

\subsection{Infrastructure Requirements}

\subsubsection{Power Systems}
\begin{itemize}
    \item Solar Array: REC Alpha Pure Series (3.52kW total)
    \item Primary Battery: Tesla Powerwall+ (13.5kWh)
    \item Backup Battery: LG RESU Prime 16H (16kWh)
    \item Controller: Victron SmartSolar MPPT 250/100
\end{itemize}

\subsubsection{Housing and Protection}
\begin{itemize}
    \item Modified shipping container ($\sim$\$5,000)
    \item Roof-mounted solar panels
    \item Internal battery storage
    \item Climate control system
\end{itemize}

\section{Power Distribution and Operation}

\subsection{Daily Power Usage}
\begin{table}[h]
\begin{tabularx}{\textwidth}{|X|c|X|}
\hline
\textbf{Time} & \textbf{Power Draw} & \textbf{Operations} \\
\hline
06:00-08:00 & 400W & System startup, calibration \\
08:00-16:00 & 600W & Maintenance, standby \\
16:00-18:00 & 800W & Evening preparation \\
18:00-22:00 & 1200W & Prime observation \\
22:00-02:00 & 900W & Standard observation \\
02:00-06:00 & 800W & Late observation \\
\hline
\end{tabularx}
\end{table}

\subsection{Emergency Power Mode}
\begin{itemize}
    \item Essential systems: 400W
    \item Security systems: 100W
    \item Computing/Control: 200W
    \item Communications: 50W
    \item Total Emergency Draw: 750W
\end{itemize}

\section{Maintenance Schedule}

\subsection{Regular Maintenance}
\begin{itemize}
    \item Daily: System checks, data backup
    \item Weekly: Solar system inspection, dust monitoring
    \item Monthly: Detailed cleaning, diagnostics
    \item Quarterly: Deep cleaning, system analysis
    \item Annual: Comprehensive review, certification
\end{itemize}

\subsection{Environmental Management}
\begin{itemize}
    \item Dust mitigation protocols
    \item Temperature monitoring
    \item Weather protection systems
    \item Emergency shutdown procedures
\end{itemize}

\section{Data Management and Access}

\subsection{Network Infrastructure}
\begin{itemize}
    \item Fiber optic connection
    \item High-speed data transfer
    \item Remote operation capability
    \item Public database access
\end{itemize}

\subsection{Data Products}
\begin{itemize}
    \item Raw observational data
    \item Processed images
    \item Calibration files
    \item Public access portal
\end{itemize}

\section{Budget and Timeline}

\subsection{Cost Breakdown}
\begin{itemize}
    \item Telescope system
    \item Infrastructure development
    \item Solar power system
    \item Security measures
    \item Transportation and installation
    \item Initial operational costs
\end{itemize}

\subsection{Implementation Timeline}
\begin{itemize}
    \item Planning and approval: 3 months
    \item Equipment procurement: 6 months
    \item Site preparation: 4 months
    \item Installation and testing: 3 months
    \item Commissioning: 2 months
\end{itemize}

\section{Conclusion}
This project will fill a critical gap in southern hemisphere astronomical observations by providing high-cadence optical monitoring capabilities. The facility's location at PARQUE ASTRONÓMICO ATACAMA, combined with its automated operations and public data access, will make it a valuable resource for the global astronomical community.

\appendix
\section{Technical Specifications}
[Detailed specifications for all equipment]

\section{Site Analysis}
[Detailed environmental and astronomical conditions]

\section{Emergency Procedures}
[Complete emergency protocols and contacts]

\end{document}